\documentclass[11pt]{article}
\usepackage[T1]{fontenc}
\usepackage[utf8]{inputenc}
\usepackage[USenglish,british,american,australian,english]{babel}
\usepackage{graphicx}
\usepackage{hyperref}
\usepackage{amsmath}
\usepackage{listings}
\usepackage{xcolor}
\lstset{
	basicstyle=\fontsize{10}{11}\ttfamily\color{black},
	commentstyle=\ttfamily\color{red},
	keywordstyle=\ttfamily\color{blue},
	stringstyle=\color{orange},
	tabsize=2,
	numbers=left,
	numberstyle=\tiny,
	firstnumber=1,
	numberfirstline=false,
	frame=single,
	showstringspaces=false,
	inputencoding=utf8,
	breaklines=true,
	language=python,
}

\title{\textbf{Weather image classification} \\ \bigskip \large Homework 2 - Machine Learning \\ Engineering in Computer Science \\ "La Sapienza" University of Rome}
\author{Costa Marco 1691388}
\date{\today}

\begin{document}
\maketitle
\pagebreak
\tableofcontents
\pagebreak

\section{Introduction}
The purpose of this homework is to solve a \textit{multi-class} image classification problem on the \textbf{\textit{Multi-class Weather Image Dataset}}. The problem must be solved using two modes:
\begin{enumerate}
	\item Define a CNN and train it from scratch;
	\item Apply transfer learning and fine tuning from a pre-trained model.
\end{enumerate}
After that, we have to evaluate the two models in a proper way. The metrics used for the comparison of the models are the accuracy and the loss function (\textit{Categorical Crossentropy}).
Finally we have to perform a prediction on the \textbf{\textit{WeatherBlindTestSet}} and create a \textit{.csv} file in wichh there is the predicted label for each image.
Moreover, we have to provide a personal dataset made with own photos. 


\subsection{Dataset}
The \textbf{\textit{Multi-class Weather Image Dataset}} contains images grouped in four classes: \textbf{HAZE}, \textbf{RAINY}, \textbf{SNOWY}, \textbf{SUNNY}.
It is very balanced: the whole dataset has 1000 images for each class.
It is organized in a file system structure with four folders corresponding to the four classes \textit{HAZE}, \textit{RAINY}, \textit{SNOWY}, \textit{SUNNY} and containing the corresponding images.
In addition, there is the \textbf{\textit{SMART-I weather test set}} that can be used as test set. It contains only the three classes \textit{RAINY}, \textit{SNOWY}, \textit{SUNNY}. 
Images in the datasets have different shapes and resolutions. It is necessary to reshape these images to make image dimensions consistent with the input layer of the network.

\subsection{Hardware}
Since we use tensorflow and keras libraries, the performance (training time, prediction, etc.) depends on the hardware, and in particular on the GPU. We used the \textit{Nvidia GeForce GTX 950M}, having 640 \textit{CUDA cores}, 4Gb of \textit{DDR3 Memory} and 1000 MHz of clock.

\section{Define a CNN and train it from scratch}
\subsection{AlexNet}


\subsection{MyNet v.1}


\subsection{MyNet v.2}


\subsection{MyNet v.3}


\section{Transfer Learning with fine tuning}


\section{Results}


\section{Conclusions}

\end{document}